\section*{Lab Exercise 5: Temperature Converter UI using React (Functional Components + useState)}

\subsection*{Question}
Design a UI for building a temperature web application that can convert Celsius to Fahrenheit and has the option to increase or decrease the temperature.

\begin{itemize}
    \item Identify the main React components in this design.
    \item For each component, list the props that would be required.
    \item Draw a component hierarchy diagram.
    \item Ensure your design uses only functional components, props, and the \texttt{useState} hook.
\end{itemize}



\subsection*{Design Description}
This React application demonstrates the use of functional components, props, and \texttt{useState} to handle temperature data.  
The UI includes:
\begin{itemize}
    \item A main temperature display showing the current value in Celsius and Fahrenheit.
    \item Buttons to increase or decrease the temperature dynamically.
    \item Real-time conversion between Celsius and Fahrenheit.
\end{itemize}

\subsection*{Main Components and Props}

\begin{itemize}
    \item \textbf{App.jsx} - Root component controlling overall logic and maintaining temperature state via \texttt{useState}.
    \begin{itemize}
        \item \textit{State:} \texttt{temperatureC} (stores current temperature in Celsius)
    \end{itemize}
    
    \item \textbf{TemperatureDisplay.jsx} - Displays the current temperature in both Celsius and Fahrenheit.
    \begin{itemize}
        \item \textit{Props:} \texttt{temperatureC}, \texttt{temperatureF}
    \end{itemize}
    
    \item \textbf{TemperatureInC.jsx} - Renders the temperature value in Celsius.
    \begin{itemize}
        \item \textit{Props:} \texttt{value}
    \end{itemize}

    \item \textbf{TemperatureInF.jsx} - Converts and renders the Fahrenheit value.
    \begin{itemize}
        \item \textit{Props:} \texttt{value}
    \end{itemize}

    \item \textbf{TemperatureControls.jsx} - Provides UI buttons to increment or decrement the temperature.
    \begin{itemize}
        \item \textit{Props:} \texttt{onIncrease}, \texttt{onDecrease}
    \end{itemize}
\end{itemize}



\subsection*{Component Hierarchy Diagram}
\begin{figure}[H]
\centering
\hspace*{-2.55cm} % <-- shift entire diagram left
\begin{tikzpicture}[
    scale=0.4,
    node distance=1.2cm and 2.2cm,
    every node/.style={
        draw,
        rounded corners,
        align=center,
        font=\footnotesize,
        fill=gray!10,
        minimum width=3.8cm,
        minimum height=0.9cm
    },
    arrow/.style={-{Latex[length=2.5mm,width=1.5mm]}, thick},
    prop/.style={font=\scriptsize, midway, fill=white, inner sep=1pt}
]

\node (app) {App.jsx\\\textit{uses useState for temperature}};
\node (display) [below left=of app, xshift=-0.5cm] {TemperatureDisplay.jsx\\\textit{props: \texttt{\{temperature, onChange\}}}};
\node (controls) [below right=of app, xshift=0.5cm] {TemperatureControls.jsx\\\textit{props: \texttt{\{increase, decrease\}}}};
\node (inf) [below=2.5cm of app] {TemperatureInF.jsx\\\textit{props: \texttt{\{celcius\}}}};
\node (inc) [below=of inf] {TemperatureInC.jsx\\\textit{no props}};

\draw[arrow] (app) -- node[prop,left]{passes temp + setter} (display);
\draw[arrow] (app) -- node[prop,right]{passes inc/dec handlers} (controls);
\draw[arrow] (app) -- node[prop,right]{passes celcius value} (inf);
\draw[arrow] (inf) -- node[prop,right]{renders Celsius block} (inc);

\end{tikzpicture}
\caption{Component Hierarchy — Lab 5: Temperature Converter (Props and Data Flow)}
\end{figure}

\subsection*{Rendered Output (Screenshot)}
\begin{figure}[H]
    \captionsetup{labelformat=empty}
    \centering
    \includegraphics[width=\textwidth]{Lab5/1.jpg}
    \caption{Screenshot 1: Thermostat UI with Temperature Controls}
\end{figure}


\subsection*{Code Overview}

\subsubsection*{App.css}
\lstinputlisting[style=codestyle, language=CSS]{Lab5/App.css}

\subsubsection*{App.jsx}
\lstinputlisting[style=codestyle, language=JavaScript]{Lab5/App.jsx}

\subsubsection*{TemperatureDisplay.jsx}
\lstinputlisting[style=codestyle, language=JavaScript]{Lab5/components/TemperatureDisplay.jsx}

\subsubsection*{TemperatureControls.jsx}
\lstinputlisting[style=codestyle, language=JavaScript]{Lab5/components/TemperatureControls.jsx}

\subsubsection*{TemperatureInC.jsx}
\lstinputlisting[style=codestyle, language=JavaScript]{Lab5/components/TemperatureInC.jsx}

\subsubsection*{TemperatureInF.jsx}
\lstinputlisting[style=codestyle, language=JavaScript]{Lab5/components/TemperatureInF.jsx}



\subsection*{Result}
The temperature converter application was successfully built using React functional components, props, and the \texttt{useState} hook.  
Users can increase or decrease the temperature dynamically, with real-time conversion between Celsius and Fahrenheit.  
The component design follows a clean hierarchy, enabling scalability and reusability.



\subsection*{Conclusion}
This lab demonstrates effective use of \texttt{useState} for dynamic UI updates and prop-based communication between parent and child components.  
It reinforces understanding of unidirectional data flow and modular UI design in React.

\subsection*{GitHub Repository}
The source code for all the labs can be found at the following GitHub repository:
\url{https://github.com/ashvp/Web-Technologies---Semester-5}
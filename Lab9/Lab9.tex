\section*{Lab Exercise 9: Basic Blog Server using Node.js HTTP Module}

\subsection*{Question}
Build a simple Node.js blog server using only the built-in \texttt{http} module by creating a \texttt{server.js} file. Implement basic routing as follows:

\begin{itemize}
    \item A \textbf{GET endpoint} at \texttt{/} that returns an HTML home page with inline navigation links to “Home” (\texttt{/}) and “Posts” (\texttt{/posts}).
    \item A \textbf{GET endpoint} at \texttt{/posts} that dynamically generates an HTML page displaying 3-4 blog posts (titles and descriptions) from a hardcoded array using a loop.
    \item A \textbf{404 handler} for unmatched routes, returning a styled HTML error message.
\end{itemize}



\subsection*{Design Description}
This lab demonstrates the creation of a minimal web server in Node.js without using external frameworks like Express.  
The server handles routing manually using conditional logic on the request URL, serving different pages based on the endpoint.

The implementation includes:
\begin{itemize}
    \item \textbf{Home Page:} Static HTML file with navigation links to other routes.
    \item \textbf{Posts Page:} Dynamically generated HTML content using a hardcoded JavaScript array of post objects.
    \item \textbf{Error Handling:} Custom 404 response for undefined routes.
\end{itemize}



\subsection*{Main Files and Their Roles}

\begin{itemize}
    \item \textbf{server.js} - Entry point of the Node application. Creates an HTTP server, handles routing logic, and serves content.
    \item \textbf{components/homePage.html} - Static home page served at the root endpoint (\texttt{/}).
    \item \textbf{components/blog.html} - Template HTML structure used to render blog posts dynamically.
    \item \textbf{components/error.html} - HTML error page returned for invalid routes.
    \item \textbf{data/BlogPosts.js} - Exports a hardcoded array of blog post objects, each containing \texttt{title} and \texttt{description}.
    \item \textbf{routes/home.js} - Handles logic for rendering and returning the home page.
    \item \textbf{routes/posts.js} - Handles logic for looping through the post data and generating the blog page.
    \item \textbf{routes/error.js} - Exports a function to handle 404 errors gracefully.
\end{itemize}



% \subsection*{Component Hierarchy Diagram}
% \begin{center}
% \includegraphics[width=0.85\textwidth]{Lab9/images/hierarchy.png}
% \end{center}



\subsection*{Rendered Output (Screenshots)}
\begin{figure}[H]
    \captionsetup{labelformat=empty}
    \centering
    \includegraphics[width=\textwidth]{Lab9/1.png}
    \caption{Screenshot 1: Home Page}
\end{figure}

\begin{figure}[H]
    \captionsetup{labelformat=empty}
    \centering
    \includegraphics[width=\textwidth]{Lab9/2.png}
    \caption{Screenshot 2: Posts Page with Dynamic Blog Posts}
\end{figure}

\begin{figure}[H]
    \captionsetup{labelformat=empty}
    \centering
    \includegraphics[width=\textwidth]{Lab9/3.png}
    \caption{Screenshot 3: Custom 404 Error Page - Invalid Route}
\end{figure}

\subsection*{Code Overview}

\subsubsection*{server.js}
\lstinputlisting[style=codestyle, language=JavaScript]{Lab9/server.js}

\subsubsection*{routes/home.js}
\lstinputlisting[style=codestyle, language=JavaScript]{Lab9/routes/home.js}

\subsubsection*{routes/posts.js}
\lstinputlisting[style=codestyle, language=JavaScript]{Lab9/routes/posts.js}

\subsubsection*{routes/error.js}
\lstinputlisting[style=codestyle, language=JavaScript]{Lab9/routes/error.js}

\subsubsection*{data/BlogPosts.js}
\lstinputlisting[style=codestyle, language=JavaScript]{Lab9/data/BlogPosts.js}

\subsubsection*{components/homePage.html}
\lstinputlisting[style=codestyle, language=HTML]
{Lab9/components/homePage.html}

\subsubsection*{components/blog.html}
\lstinputlisting[style=codestyle, language=HTML]{Lab9/components/blog.html}

\subsubsection*{components/error.html}
\lstinputlisting[style=codestyle, language=HTML]{Lab9/components/error.html}



\subsection*{Result}
The Node.js HTTP server was successfully implemented with basic routing and dynamic content generation.  
Navigating to \texttt{http://localhost:3000/} renders the home page, while \texttt{/posts} dynamically displays 3-4 blog posts from the array.  
Invalid routes correctly return a custom 404 page.



\subsection*{Conclusion}
This lab demonstrates how to create a functional web server using only Node.js’ built-in \texttt{http} and \texttt{fs} modules.  
It showcases:
\begin{itemize}
    \item Manual routing without frameworks.
    \item Dynamic HTML generation using JavaScript loops.
    \item Modular separation of route handlers and data sources.
\end{itemize}
The project lays the foundation for understanding core Node.js web concepts before introducing Express.js in advanced labs.

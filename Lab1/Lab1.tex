\lstset{inputpath=.} 

\section*{Lab Exercise 1: Resume Webpage using HTML and CSS}

\subsection*{Question}
Create an HTML and CSS webpage that captures a resume (preferably yours) for a fresher full-stack developer at an IT company. Decide on the subdivisions and content. Push your content to GitHub and maintain proper version control with clear README files.

\subsection*{Design / Plan}
The webpage is designed with semantic HTML5 elements for clarity and SEO-friendly structure.  
It consists of the following sections:
\begin{itemize}
    \item Header - Name and tagline
    \item About Me - Short profile summary
    \item Skills - Categorized technical stack
    \item Projects - Portfolio projects with GitHub/Live links
    \item Education - Academic background
    \item Experience - Internship and professional exposure
    \item Certifications - Relevant courses and achievements
    \item Footer - Contact and social links
\end{itemize}

CSS provides a minimalistic, modern aesthetic with consistent padding, section styling, and accent colors.

\subsection*{Code}

\subsubsection*{HTML (resume.html)}
\lstinputlisting[style=codestyle, language=HTML]{Lab1/resume.html}

\newpage
\subsubsection*{CSS (styleResume.css)}
\lstinputlisting[style=codestyle, language=CSS]{Lab1/styleResume.css}

\subsection*{Screenshots}

\begin{figure}[H] % You can use [H] to force the figure to stay here, or [htbp] for more flexible positioning
    \centering
    \includegraphics[width=\textwidth]{Lab1/1.png}
    \caption*{Screenshot 1: Introduction and Skills}
    \label{fig:screenshot1}
\end{figure}

\begin{figure}[H]
    \centering
    \includegraphics[width=\textwidth]{Lab1/2.png}
    \caption*{Screenshot 2: Projects and Education}
    \label{fig:screenshot2}
\end{figure}

\begin{figure}[H]
    \centering
    \includegraphics[width=\textwidth]{Lab1/3.png}
    \caption*{Screenshot 3: Experience and Certifications}
    \label{fig:screenshot3}
\end{figure}


\subsection*{Result}
The HTML and CSS files successfully create a responsive, professional resume webpage for a fresher full-stack developer.  
All components render properly in Chrome and Edge browsers. The webpage was version-controlled and pushed to GitHub for public access.

\vspace{3cm}

\section*{Question 2: CV Webpage using HTML and CSS}

\subsection*{Question}
Create an HTML and CSS webpage that captures a CV of a fresher (preferably yours). Decide on the subdivisions and content. Push your content to GitHub and maintain proper version control with clear Read Me files.

\subsection*{Design / Plan}
The CV webpage extends the previous resume layout by including a more detailed representation of the candidate’s profile.  
It uses a section-based structure for clear readability and modern minimal design.  

Key subdivisions include:
\begin{itemize}
    \item \textbf{Header} - Name, title, and tagline.
    \item \textbf{About Me} - Short summary about the developer's goals and interests.
    \item \textbf{Skills} - Categorized stack (Frontend, Backend, Databases, Tools).
    \item \textbf{Projects} - Showcase of major projects with live/GitHub links.
    \item \textbf{Education} - Academic details (B.Tech, B.S, 10th/12th).
    \item \textbf{Experience} - Internship details with specific contributions.
    \item \textbf{Certifications} - Relevant technical achievements.
    \item \textbf{Extra-Curricular} - Sports, achievements, leadership.
    \item \textbf{Footer} - Contact and online profiles.
\end{itemize}

The CSS maintains uniform padding, typography, and color contrast for clarity.  

\subsection*{Code}

\subsubsection*{HTML (cv.html)}
\lstinputlisting[style=codestyle, language=HTML]{Lab1/cv.html}

\subsubsection*{CSS (styleCV.css)}
\lstinputlisting[style=codestyle, language=CSS]{Lab1/styleCV.css}

\subsection*{Screenshots}

\begin{figure}[H] % You can use [H] to force the figure to stay here, or [htbp] for more flexible positioning
    \centering
    \includegraphics[width=\textwidth]{Lab1/4.png}
    \caption*{Screenshot 1: Introduction and Skills}
    \label{fig:screenshot1}
\end{figure}

\begin{center}
    \includegraphics[width=\textwidth]{Lab1/5.png}\\\    \textbf{Screenshot 2: Projects}
\end{center}
\vspace{1em}

\begin{figure}[H]
    \centering
    \includegraphics[width=\textwidth]{Lab1/6.png}
    \caption*{Screenshot 3: Education and Experience}
    % \label{fig:screenshot3}
\end{figure}

\begin{figure}[H]
    \centering
    \includegraphics[width=\textwidth]{Lab1/7.png}
    \caption*{Screenshot 4: Certifications and Extra-Curricular Activities}
    % \label{fig:screenshot3}
\end{figure}

\subsection*{Result}
The HTML and CSS files successfully produce a well-structured, aesthetically pleasing CV webpage for a fresher full-stack developer.  
The design clearly separates sections, maintaining a clean and professional look suitable for digital submission.  
All elements render properly in Chrome and Edge browsers.  


\subsection*{GitHub Repository}
The source code for all the labs can be found at the following GitHub repository:
\url{https://github.com/ashvp/Web-Technologies---Semester-5}
\section*{Lab Exercise 2: Weather Dashboard using HTML, CSS, and JavaScript}

\subsection*{Question}
Develop a weather dashboard that fetches and displays current weather data for a city using a free public API.

\textbf{Requirements:}
\begin{itemize}
    \item Define the essential data points to display (e.g., temperature, city name, weather icon, humidity, wind speed, etc.).
    \item Design a weather card UI to present the information.
    \item Use HTML for structure, CSS for styling, and JavaScript Fetch API (with \texttt{async/await}) for handling API requests and updating the DOM.
    \item Implement proper error handling for invalid city names.
    \item Host the project on GitHub with a clear README.
\end{itemize}

\subsection*{Requirement Analysis}
Essential data to display:
\begin{itemize}
    \item City name
    \item Temperature (°C)
    \item Weather condition and icon
    \item Humidity (\%)
    \item Wind speed (m/s)
\end{itemize}

External API used:
\begin{itemize}
    \item OpenWeatherMap API (\url{https://openweathermap.org/api})
    \item Free-tier key integrated via the Fetch API
\end{itemize}

\subsection*{Design}
A minimalist card-style UI is designed to display real-time weather information.  
The core layout includes:
\begin{itemize}
    \item A header and footer with neutral tones
    \item A centered search bar for entering the city name
    \item A card that dynamically shows weather data fetched via JavaScript
    \item Color-coded error message display
\end{itemize}

A design sketch (in Figma) illustrates modular separation between input, display, and data logic layers.

\subsection*{Code}

\subsubsection*{HTML (index.html)}
\lstinputlisting[style=codestyle, language=HTML]{Lab2/index.html}

\subsubsection*{CSS (style.css)}
\lstinputlisting[style=codestyle, language=CSS]{Lab2/style.css}

\subsubsection*{JavaScript (script.js)}
\lstinputlisting[style=codestyle, language=JavaScript]{Lab2/script.js}

\begin{figure}[H]
    \captionsetup{labelformat=empty}
    \centering
    \includegraphics[width=\textwidth]{Lab2/1.png}
    \caption{Screenshot 1: Initial Page}
\end{figure}

\begin{figure}[H]
    \captionsetup{labelformat=empty}
    \centering
    \includegraphics[width=\textwidth]{Lab2/2.png}
    \caption{Screenshot 2: Weather for Chennai}
\end{figure}

\begin{figure}[H]
    \captionsetup{labelformat=empty}
    \centering
    \includegraphics[width=\textwidth]{Lab2/3.png}
    \caption{Screenshot 3: Error Handling}
\end{figure}

\subsection*{Testing}
The application was tested for:
\begin{itemize}
    \item \textbf{Valid Input:} Correctly displays weather for cities like \texttt{Chennai}, \texttt{Delhi}, \texttt{London}.
    \item \textbf{Invalid Input:} Displays clear error message when city not found.
    \item \textbf{Network Failure:} Graceful error handling with message.
    \item \textbf{Edge Case:} Tested empty input and special characters.
\end{itemize}

All test cases passed as expected.

\subsection*{Deployment and Version Control}
The project was version-controlled with Git and pushed to GitHub.  
Each commit represents one functional milestone: UI design, API integration, error handling, and deployment.

Repository URL:  
\begin{center}
\href{https://github.com/ashvp/weather-dashboard}{\texttt{https://github.com/ashvp/weather-dashboard}}
\end{center}

\subsection*{Result}
A fully functional weather dashboard web application was developed using HTML, CSS, and JavaScript.  
It successfully fetches and displays weather data for any valid city using OpenWeatherMap’s public API, and handles all error scenarios gracefully.  
The UI is responsive, accessible, and visually consistent across browsers.

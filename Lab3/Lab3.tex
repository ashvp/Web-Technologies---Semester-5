\section*{Lab Exercise 3: Event Registration Form with Validation using HTML, CSS, and JavaScript}

\subsection*{Question}
Develop a registration form that validates user input for a community event signup using HTML, CSS, and JavaScript.

\textbf{Requirements:}
\begin{itemize}
    \item Define all form fields and validation requirements (e.g., name, email, phone number, age).
    \item Specify which fields are required and define valid formats (regex for email, digits for phone number, age over 18, etc.).
    \item Design a user-friendly registration UI that provides real-time feedback for invalid fields.
    \item Implement the form using HTML, CSS, and JavaScript (with real-time validation).
    \item Test for invalid inputs and handle edge cases.
    \item Host and version-control the project on GitHub.
\end{itemize}

\subsection*{Requirement Analysis}
The form includes the following input fields:
\begin{itemize}
    \item \textbf{Full Name} - Required, must not be empty.
    \item \textbf{Email Address} - Required, must match regex \verb|/^\S+@\S+\.\S+$/|.
    \item \textbf{Phone Number} - Required, must contain exactly 10 digits.
    \item \textbf{Age} - Required, must be a number greater than or equal to 18.
\end{itemize}

Validation logic ensures:
\begin{itemize}
    \item Real-time error detection and display.
    \item Form submission disabled until all inputs are valid.
    \item Immediate visual feedback (green border for valid, red for invalid).
\end{itemize}

\subsection*{Design}
The registration form is designed with a centered card layout and clear label-input pairs.  
Design highlights:
\begin{itemize}
    \item A clean, modern interface with white background and subtle shadows.
    \item Red and green border indicators for validation.
    \item Error messages positioned directly below each field.
    \item Submit button dynamically enabled only when all inputs are valid.
\end{itemize}

\subsection*{Code}

\subsubsection*{HTML (index.html)}
\lstinputlisting[style=codestyle, language=HTML]{Lab3/index.html}

\subsubsection*{CSS (style.css)}
\lstinputlisting[style=codestyle, language=CSS]{Lab3/style.css}

\subsubsection*{JavaScript (script.js)}
\lstinputlisting[style=codestyle, language=JavaScript]{Lab3/script.js}

\begin{figure}[H]
    \captionsetup{labelformat=empty}
    \centering
    \includegraphics[width=\textwidth]{Lab3/1.png}
    \caption{Screenshot 1: Initial Page}
\end{figure}

\begin{figure}[H]
    \captionsetup{labelformat=empty}
    \centering
    \includegraphics[width=\textwidth]{Lab3/2.png}
    \caption{Screenshot 2: Frontend Data Validation}
\end{figure}

\begin{figure}[H]
    \captionsetup{labelformat=empty}
    \centering
    \includegraphics[width=\textwidth]{Lab3/3.png}
    \caption{Screenshot 3: Validated Inputs}
\end{figure}

\begin{figure}[H]
    \captionsetup{labelformat=empty}
    \centering
    \includegraphics[width=\textwidth]{Lab3/4.png}
    \caption{Screenshot 4: Popup to Accept Registration}
\end{figure}

\subsection*{Testing}
Test cases were designed to verify both functional and boundary behavior:

\begin{itemize}
    \item \textbf{Empty Fields:} Proper error messages shown for missing input.
    \item \textbf{Invalid Email:} Input without '@' or domain suffix triggers error.
    \item \textbf{Invalid Phone:} Less or more than 10 digits rejected.
    \item \textbf{Age Validation:} Users below 18 cannot submit.
    \item \textbf{Valid Input:} All validations pass, “Sign Up” button becomes active.
\end{itemize}

\textbf{Edge Cases:}
\begin{itemize}
    \item Input with spaces only in name field.
    \item Mixed letters in phone number.
    \item Decimal or negative values in age field.
\end{itemize}

All invalid entries trigger immediate visual and textual feedback, and submission is blocked until fixed.

\subsection*{Deployment and Version Control}
\begin{itemize}
    \item Project maintained in GitHub repository with modular commits for HTML, CSS, JS, and validation logic.
    \item README file documents validation rules, test scenarios, and setup instructions.
    \item The form is deployed live using GitHub Pages.
\end{itemize}

Repository URL:
\begin{center}
\href{https://github.com/ashvp/event-registration-form}{\texttt{https://github.com/ashvp/event-registration-form}}
\end{center}

\subsection*{Result}
A responsive, interactive registration form was successfully developed and validated using HTML, CSS, and JavaScript.  
All validations, error handling, and edge cases were implemented as per requirements.  
The UI offers a seamless and intuitive experience, preventing invalid submissions and ensuring clean data entry for event signups.

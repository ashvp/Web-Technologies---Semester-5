\section*{Lab Exercise 6: Countdown Timer Application using React}

\subsection*{Question}
Design a UI for a countdown timer web application where users can set the time and start/stop the countdown.

\begin{itemize}
    \item Identify and define the main React components (\texttt{Title}, \texttt{TimeSetter}, \texttt{TimerDisplay}, \texttt{ControlButtons}).
    \item List the props required for each component.
    \item Draw a component hierarchy diagram.
    \item Use functional components, \texttt{props}, \texttt{useState}, and \texttt{useEffect} for managing timer intervals.
\end{itemize}



\subsection*{Design Description}
This React application provides an interactive countdown timer interface that allows users to:
\begin{itemize}
    \item Set a custom countdown duration.
    \item Start, pause, and reset the timer dynamically.
    \item View the remaining time in real-time.
\end{itemize}

The app uses React's \texttt{useState} for managing timer data and \texttt{useEffect} for interval-based countdown logic.  
Each component is functional, modular, and communicates through props.



\subsection*{Main Components and Props}

\begin{itemize}
    \item \textbf{App.jsx} - The parent component that manages state and interval logic.  
    \textit{State variables:} \texttt{timeLeft}, \texttt{isRunning}.

    \item \textbf{Title.jsx} - Displays the main title or heading for the application.  
    \textit{Props:} \texttt{text} (string).

    \item \textbf{TimeSetter.jsx} - Allows the user to input or adjust the countdown duration.  
    \textit{Props:} \texttt{onTimeChange} (function to update timer duration), \texttt{disabled} (boolean to disable input during active countdown).

    \item \textbf{TimerDisplay.jsx} - Displays the formatted countdown (minutes:seconds).  
    \textit{Props:} \texttt{timeLeft} (integer representing total seconds).

    \item \textbf{ControlButtons.jsx} - Contains Start, Pause, and Reset buttons to control the countdown timer.  
    \textit{Props:} \texttt{isRunning} (boolean), \texttt{onStart}, \texttt{onPause}, \texttt{onReset}.
\end{itemize}



\subsection*{Component Hierarchy Diagram}
\begin{figure}[H]
\centering
\hspace*{-1cm} % adjust if cutting on the right
\begin{tikzpicture}[
    node distance=2cm and 2cm,
    every node/.style={
        draw,
        rounded corners,
        align=center,
        font=\footnotesize,
        fill=gray!10,
        minimum width=1cm,
        minimum height=1cm,
        text width=3.8cm
    },
    arrow/.style={-{Latex[length=2.5mm,width=1.5mm]}, thick},
    prop/.style={font=\scriptsize, midway, fill=white, inner sep=1pt}
]

%  Parent node 
\node (app) {App.jsx\
\textit{uses useState/useEffect for timer logic}};

%  Child nodes (fan layout) 
\node (title) [below left=3cm and 3cm of app] {Title.jsx\
\textit{props: \texttt{\symbol{123}text\symbol{125}}}};
\node (setter) [below left=6cm and 2cm of app] {TimeSetter.jsx\
\textit{props: \texttt{\symbol{123}hours, minutes, seconds, onTimeChange, disabled\symbol{125}}}};
\node (controls) [below right=6cm and -3cm of app] {ControlButtons.jsx\
\textit{props: \texttt{\symbol{123}isActive, onStart, onStop, onReset, hasTime\symbol{125}}}};
\node (display) [below right=3cm and 3cm of app] {TimerDisplay.jsx\
\textit{props: \texttt{\symbol{123}hours, minutes, seconds, isActive, isFinished\symbol{125}}}};

%  Arrows 
\draw[arrow] (app) -- node[prop,above left, pos=0.5]{passes title text} (title);
\draw[arrow] (app) -- node[prop,right, pos=0.65]{passes setters + disable state} (setter);
\draw[arrow] (app) -- node[prop,right, pos=0.45]{passes event handlers} (controls);
\draw[arrow] (app) -- node[prop,above right, pos=0.55]{passes timer values + flags} (display);

\end{tikzpicture}
\caption{Component Hierarchy — Lab 6: Countdown Timer (Fan-out Structure with Prop Flow)}
\end{figure}




\subsection*{Rendered Output (Screenshot)}
\begin{figure}[H]
    \captionsetup{labelformat=empty}
    \centering
    \includegraphics[width=\textwidth]{Lab6/1.png}
    \caption{Screenshot 1: Countdown Timer UI}
\end{figure}

\begin{figure}[H]
    \captionsetup{labelformat=empty}
    \centering
    \includegraphics[width=\textwidth]{Lab6/2.png}
    \caption{Screenshot 2: Timer Running}
\end{figure}
\begin{figure}[H]
    \captionsetup{labelformat=empty}
    \centering
    \includegraphics[width=\textwidth]{Lab6/3.jpg}
    \caption{Screenshot 3: Timer Finished}
\end{figure}

\newpage
\subsection*{Code Overview}

\subsubsection*{App.css}
\lstinputlisting[style=codestyle, language=CSS]{Lab6/App.css}

\subsubsection*{App.jsx}
\lstinputlisting[style=codestyle, language=JavaScript]{Lab6/App.jsx}

\subsubsection*{Title.jsx}
\lstinputlisting[style=codestyle, language=JavaScript]{Lab6/components/Title.jsx}

\subsubsection*{TimeSetter.jsx}
\lstinputlisting[style=codestyle, language=JavaScript]{Lab6/components/TimeSetter.jsx}

\subsubsection*{TimerDisplay.jsx}
\lstinputlisting[style=codestyle, language=JavaScript]{Lab6/components/TimerDisplay.jsx}

\subsubsection*{ControlButtons.jsx}
\lstinputlisting[style=codestyle, language=JavaScript]{Lab6/components/ControlButtons.jsx}



\subsection*{Result}
The countdown timer web application was successfully designed and implemented using React functional components.  
The timer logic was managed using \texttt{useState} and \texttt{useEffect} hooks to update the countdown in real-time.  
The design supports modular component-based architecture with clear prop communication.



\subsection*{Conclusion}
This exercise demonstrates effective use of React hooks for managing side effects (intervals) and dynamic state updates.  
By splitting logic across four functional components, the app achieves a clean, maintainable, and scalable design.

\subsection*{GitHub Repository}
The source code for all the labs can be found at the following GitHub repository:
\url{https://github.com/ashvp/Web-Technologies---Semester-5}